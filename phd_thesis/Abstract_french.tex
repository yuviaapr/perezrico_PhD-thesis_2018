\begin{center}
	\bfseries Resum\'e
\end{center}

	La r\'egulation des g\`enes chez les vert\'ebr\'es implique une multitude de r\'egulateurs et de m\'ecanismes cellulaires destin\'es \`a orchestrer des interactions complexes contr\^olant les r\'eponses cellulaires et constituant la base de l'\'etablissement de l'identit\'e cellulaire. La r\'egulation transcriptionnelle est consid\'er\'ee comme le premier niveau pour moduler l’expression g\'enique. Pour augmenter le niveau de transcription basal des g\`enes, les promoteurs doivent interagir avec des regions g\'enomiques activatrices comme les amplificateurs. Les g\`enes qui contr\^olent l'identit\'e cellulaire sont fr\'equemment associ\'es avec des regions tr\`es riches en amplificateurs, appel\'ees super-amplificateurs. Les super-amplificateurs sont tr\`es sensibles aux changements de concentration des prot\'eines qui leur sont li\'ees et ils pr\'esentent une haute inter-connectivit\'e au niveau de la chromatine. \'Egalement, les super-amplificateurs sont souvent organis\'es dans le noyau cellulaire en tant que r\'egions isol\'ees d\'elimit\'ees par des prot\'eines architecturales, telles que le facteur de liaison \`a CCCTC (CTCF). Consid\'erant les r\'eseaux complexes que les super-amplificateurs peuvent former avec leurs g\`enes cibles, il existe l’hypoth\`ese qu’ils peuvent participer \`a la formation des agr\'egats par s\'eparation de phase. D'autre part, CTCF poss\`ede des fonctions pl\'e\"iotropiques qui d\'ependent des sites de liaison \`a l’ADN et du type d'interactions de la chromatine dans lesquelles il est impliqu\'e. En plus de conf\'erer une isolation aux r\'egions g\'enomiques des super-amplificateurs, CTCF est important pour l’isolation des domaines structurels dans les g\'enomes et pour la formation de boucles d’ADN. Pour ces raisons, les super-amplificateurs et CTCF sont consid\'er\'es comme des acteurs centraux de l'organisation tridimensionnelle du g\'enome ayant un impact direct sur la r\'egulation de l'expression g\'enique.\\

	M\^eme si les fonctions des super-amplificateurs et de CTCF ont \'et\'e largement \'etudi\'ees chez les mammif\`eres, l’analyse de conservation de leurs fonctions chez les vert\'ebr\'es g\'en\'etiquement \'eloign\'es des mammif\`eres reste largement inexplor\'ee. Ici, pour mieux comprendre leur conservation fonctionnelle, des analyses de super-amplificateurs et de CTCF chez le poisson z\`ebre ont \'et\'e effectu\'ees. Les super-amplificateurs annot\'es dans le g\'enome du poisson z\`ebre pr\'esentent une grande sp\'ecificit\'e cellulaire et tissulaire, et la principale diff\'erence identifi\'ee par rapport aux super-amplificateurs chez la souris et l’humain est leur distribution autour des sites d'initia-tion de la transcription. Des analyses de conservation indiquent que les super-amplificateurs n'ont pas une conservation de s\'equence sup\'erieure \`a celle des amplificateurs. N\'eanmoins, en limitant l'analyse de conservation de s\'equence aux super-amplificateurs situ\'es \`a proximit\'e d'orthologues chez le poisson z\`ebre, la souris et l'humain, il est possible d'identifier un sous-ensemble de super-amplificateurs ayant une conservation de s\'equence plus \'elev\'ee que le reste des super-amplificateurs. La comparaison d'expression r\'egul\'ee par les r\'egions constitutives de deux super-amplificateurs associ\'es \`a des orthologues a permis l'identification de r\'egions fonctionnellement conserv\'ees, malgr\'e l'absence de conservation \'evidente de la s\'equence d’ADN. En ce qui concerne CTCF, les analyses des r\'egions de liaison \`a CTCF situ\'ees dans les promoteurs des g\`enes indiquent une correlation entre l’abondance de CTCF et l’expression g\'enique, ce qui pourrait s’expliquer par le blocage du d\'ep\^ot de nucl\'eosomes dans ces promoteurs. Pourtant, contrairement \`a la distribution de CTCF observ\'ee autour des domaines de contact dans les g\'enomes de mammif\`eres, CTCF n’est pas enrichi aux limites des domaines identifi\'es chez le poisson z\`ebre.\\

	En r\'esum\'e, ces r\'esultats mettent en \'evidence des fonctions conserv\'ees et divergentes des r\'egulateurs de g\`enes tout au long de l'\'evolution des vert\'ebr\'es et constituent une base pour l'\'etude de l'organisation du g\'enome chez le poisson z\`ebre. L'int\'egration future des annotations de super-amplificateurs et des r\'egions de liaison \`a CTCF sera importante pour analyser la contribution de ces r\'egulateurs et d'autres r\'egulateurs dans ce processus complexe.\\



