\chapter{Annexes}

	\section{Resum\'e}

		Les vert\'ebr\'es sont des chord\'es caract\'eris\'es par la pr\'esence d'une colonne vert\'ebrale. En tant qu'organismes multicellulaires, ils repr\'esentent des mosa\"iques de types cellulaires. Comprendre quels m\'ecanismes et quels r\'egulateurs contr\^olent le destin cellulaire et conduisent \`a la vari\'et\'e de morphologies et de physiologies observ\'ees chez les vert\'ebr\'es est une des \'enigmes de la biologie. Le g\'enome est le principal point commun entre les diff\'erents types de cellules d'un organisme et cette caract\'eristique s'applique \`a presque tous les types de cellules somatiques saines, sauf quelques exceptions, comme les cellules du syst\`eme immunitaire. En cons\'equence, \'etant donn\'e qu'il n'y a pas de diff\'erences majeures en termes de contenu g\'enique entre les diverses types cellulaires, ce qui g\'en\`ere la myriade de types de cellules chez un vert\'ebr\'e est la r\'egulation pr\'ecise des g\`enes.\\

		L'expression des g\`enes chez les vert\'ebr\'es et d'autres eucaryotes est influenc\'ee par les \'ev\'ene-ments r\'egulateurs dans le noyau et dans le cytoplasme. Dans ce travail, je me suis concentr\'ee sur le premier niveau r\'egulateur de l'expression g\'enique qui est la r\'egulation transcriptionnelle. En particulier, j'ai analys\'e des m\'ecanismes et r\'egulateurs contr\^olant l'initiation et le maintien de la transcription.\\

		Les promoteurs sont d\'efinis comme les s\'equences entourant les sites d'initiation de la transcription capables de recruter des facteurs de transcription basaux et des co-activateurs pour d\'eclencher la transcription productive de leur g\`ene associ\'e, et moins fr\'equemment, pour r\'eguler l'expression de g\`enes distaux. L'expression des g\`enes est \'egalement contr\^ol\'ee par des amplificateurs, qui sont des \'el\'ements r\'egulateurs qui stimulent l'expression de leurs g\`enes cibles ind\'ependamment de leur localisation par rapport aux promoteurs. Donc, les amplificateurs peuvent interagir avec leurs promoteurs cibles localis\'es \`a de longues distances lin\'eaires. Des analyses d'amplificateurs dans les cellules souches chez la souris ont r\'ev\'el\'e que des g\`enes importants pour la sp\'ecification de la pluripotence sont \`a proximit\'e des agglom\'er\'es d'amplifica-teurs. Ces agglom\'er\'es sont d\'esign\'es sous le nom de super-amplificateurs. Cependant, les super-amplificateurs ne sont pas des r\'egulateurs restreints aux cellules souches et ils ont \'et\'e aussi identifi\'es dans d'autres types cellulaires et tissus, exposant son association pr\'ef\'erentielle avec des g\`enes d'identit\'e cellulaire. Pour avoir une compr\'ehension globale des m\'ecanismes d'action des super-amplificateurs, il est n\'ecessaire de les analyser en consid\'erant le contexte nucl\'eaire r\'eel, dans lequel les g\'enomes sont organis\'es dans un espace tridimensionnel (3D). Par exemple, les super-amplificateurs sont organis\'es dans des r\'egions nucl\'eaires isol\'es qui sont d\'elimit\'es par CTCF et le complexe prot\'eique cohesin. Notamment, il a \'et\'e d\'emontr\'e que les super-amplificateurs ont une colocalisation spatiale pour participer simultan\'ement \`a des interactions complexes avec plusieurs g\`enes cibles.\\

		Pendant l'interphase, tous les chromosomes sont pr\'ef\'erentiellement s\'epar\'es en volumes d\'efinis, appel\'es territoires chromosomiques, qui peuvent interagir avec leurs territoires voisins. De plus, les chromosomes sont compartiment\'es dans deux types de domaines en fonction de leur \'etat de chromatine et des r\'egions discr\`etes de chromatine auto-interagissant peuvent \^etre identifi\'ees dans les donn\'ees g\'en\`eres par des exp\'eriences de Hi-C en utilisant des populations de cellules. Ces r\'egions discr\`etes sont appel\'ees domaines de contact et \'etablissent des volumes confin\'es qui favorisent les interactions des g\`enes avec leurs \'el\'ements r\'egulateurs. En plus de participer \`a la formation de r\'egions nucl\'eaires de super-amplificateurs, CTCF est enrichi aux limites des domaines de contact. En effet, les boucles d'ADN m\'edi\'ees par CTCF peuvent r\'ecapituler les cartes de contact obtenues par exp\'eriences d'haute r\'esolution de Hi-C chez l'humain. Aussi, il a \'et\'e d\'etermin\'e que les ancres des domaines de contact sont principalement observ\'ees entre des r\'egions contenant des sites de CTCF dans une orientation convergente. Par cons\'equent, il existe un enrichissement des sites de CTCF divergents aux limites des domaines de contact. D\'el\'etions de ces sites de CTCF permettent la formation d'interactions ectopiques entre les amplificateurs et les promoteurs de domaines de contact adjacents. Pour ces raisons, le CTCF est consid\'er\'e comme un important r\'egulateur de l'organisation du g\'enome 3D.\\

		Cependant, la plupart de nos connaissances sur la r\'egulation des g\`enes chez les vert\'ebr\'es ont \'et\'e obtenues par des analyses du g\'enome de la souris et de l'humain. Afin de d\'eterminer si les fonctions des super-amplificateurs et de CTCF sont conserv\'ees chez un vert\'ebr\'e distant des mammif\`eres, je me suis concentr\'ee sur la caract\'erisation des super-amplificateurs et sur l'analyse de la liaison \`a l'ADN de CTCF chez le poisson z\`ebre.\\

		Des annotations de super-amplificateurs ont \'et\'e g\'en\'er\'ees sur la base de l'enrichissement du signal ChIP-seq de H3K27ac dans quatre tissus et cellules pluripotentes chez le poisson zèbre. Une forte sp\'ecificit\'e cellulaire et tissulaire a \'et\'e identifi\'ee comme un point commun entre les super-amplificateurs chez les mammifères et le poisson zèbre. L'analyse comparative de la conservation de la s\'equence des r\'egions constitutives des super-amplificateurs et du reste des amplificateurs n'indique pas une conservation de la s\'equence diff\'erentielle. N\'eanmoins, un groupe de gènes orthologues situ\'es à proximit\'e des super-amplificateurs chez le poisson zèbre, la souris et l'humain a \'et\'e identifi\'e. Les super-amplificateurs associ\'es à ces gènes montrent une conservation de s\'equence plus \'elev\'ee que ceux sans associations avec des orthologues. Aussi, des essais pour deux super-amplificateurs avec des gènes rapporteurs ont \'et\'e utilis\'es pour identifier des r\'egions avec des fonctions \'equivalentes chez le poisson zèbre et la souris.\\

		La liaison de CTCF au g\'enome du poisson z\'ebre a \'et\'e difficile \`a \'evaluer en raison de l'absence d'anticorps de haut qualit\'e. Pour surmonter cette limitation, la technologie d'\'edition des g\'enomes CRISPR / Cas9 a \'et\'e utilis\'ee pour ins\'erer une \'etiquette HA dans la s\'equence codant de \textit{ctcf}. En utilisant cette lign\'ee de poisson transg\'enique, des donn\'ees de ChIP-seq ont \'et\'e g\'en\'er\'ees. De mani\`ere similaire aux sites de liaison de CTCF d\'ecrits chez les vert\'ebr\'es pr\'ec\'edemment analys\'es, des motifs \'etendus de liaison de CTCF ont \'et\'e identifi\'es dans une fraction des sites de liaison de CTCF. En outre, des s\'equences r\'ep\'etitives enrichies sur les sites de liaison de CTCF ont \'et\'e annot\'ees, sugg\'erant quelles s\'equences pourraient contribuer \`a l'expansion de ces sites chez le poisson z\`ebre. Une association positive entre l'abondance de CTCF et l'expression g\'enique a \'et\'e identifi\'ee pour les sites de liaison de CTCF situ\'es dans les promoteurs. L'analyse de l'accessibilit\'e de l'ADN dans ces r\'egions sugg\`ere un m\'ecanisme dans lequel CTCF facilite l'\'etablissement de r\'egions exemptes de nucl\'eosomes dans les promoteurs qui favorisent des hauts niveaux d'expression. Enfin, pour confirmer le r\^ole de CTCF dans l'organisation du g\'enome chez le poisson z\`ebre, des cartes de Hi-C ont \'et\'e analys\'ees. Contrairement \`a ce qui a \'et\'e \'etablis chez les mammifères, CTCF n'est pas d\'etect\'e comme enrichi aux limites des domaines de contact chez les embryons de poisson z\`ebre, mais des marques de chromatine associ\'ees à la transcription active sont enrichies dans ces r\'egions. M\^eme si le CTCF n'est g\'en\'eralement pas enrichi aux limites de domaines, une fraction d'entre eux contient des sites de CTCF, donc, il reste \`a analyser si les motifs situ\'es dans ces r\'egions ont l'orientation pr\'ef\'er\'ee observ\'ee chez les mammif\`eres.\\

		Le principal inconv\'enient des annotations de super-amplificateur pr\'esent\'ees ici, \`a l'exception de celles des cellules pluripotentes, est que les banques de ChIP-seq utilis\'ees pour leur annotation ont \'et\'e pr\'epar\'ees par homog\'en\'eisation de tissus entiers. Cela implique que le signal identifi\'e repr\'esente la moyenne de la population de diff\'erents types cellulaires. Par cons\'equent, il est possible qu'une fraction des super-amplificateurs annot\'es corresponde \`a des faux positifs caus\'es par l'effet de la fusion d'amplificateurs sp\'ecifiques situ\'es autour de la m\^eme r\'egion dans diff\'erents types cellulaires. De m\^eme, la principale limitation \`a la caract\'erisation des fonctions de CTCF chez les embryons est l'h\'et\'erog\'en\'eit\'e des \'echantillons utilis\'es pour l'identification des sites de liaison.\\

		La strat\'egie appliqu\'ee pour \'etudier le CTCF chez le poisson z\`ebre serait utile dans l'analyse d'autres facteurs de transcription. Aussi, la ligne \'etablie du poisson z\`ebre permettrait de mener des analyses plus cibl\'ees sur l'organisation du g\'enome afin de caract\'eriser en profondeur les fonctions de CTCF.\\

		En conclusion, dans ce travail, j'ai analys\'e les caract\'eristiques et les fonctions des super-amplificateurs et de CTCF chez le poisson z\`ebre. Les r\'esultats obtenus ici seront pr\'ecieux pour concevoir des strat\'egies permettant d'\'evaluer plus en profondeur les fonctions de ces r\'egulateurs et leur impact sur l'organisation du g\'enome 3D. L'int\'egration des annotations de super-amplificateurs et sites de liaison de CTCF sera importante pour identifier d'autres r\'egulateurs du g\'enome et ses fonctions sp\'ecifiques chez le poisson z\`ebre.\\

	\newpage

	\section{MicroRNA degradation by a conserved target RNA regulates animal behavior}

		miRNAs are small noncoding RNAs that regulate gene expression at the post-transcriptional level by a mechanism based on the pairing of miRNA positions 2-8 to their target RNAs. \textit{In vitro} analyses in cell lines using exogenous RNAs with extensive complementarity to miRNAs have previously shown that increase in the pairing between miRNAs and target sequences can lead to the degradation of miRNAs. However, endogenous RNA targets regulating the levels of miRNAs had not been identified. In this article, the first example of endogenous miRNA degration in animals is reported. First, \textit{libra}, a long noncoding RNA containing a highly conserved near-perfect miR-29 binding site, was characterized in zebrafish. \textit{libra} expression is enriched in brain tissues, and disruption of the genomic locus in zebrafish results in behavioral abnormalities. Similarly, mutant mice carrying a scrambled miR-29 binding site in the mouse ortholog of \textit{libra} (\textit{Nrep}) have motor learning impairment. This is concordant with the gain of ectopic expression of mir-29b in mouse cerebellum. Analyses of mature miRNAs in mouse neural progenitor cells show accumulation of mir-29b in mutant cells with the scrambled miR-29 binding site in \textit{Nrep}, while trimmed versions of mir-29b were identified in wild type cells. In conclusion, these results support that \textit{Nrep} regulates mir-29b levels by target RNA-directed miRNA degradation.\\

		\subsection{Contribution}

			My contribution to this study focused on the evolutionary analyses of \textit{libra} in vertebrates. It was previously reported that \textit{libra} contained 3 blocks of high sequence conservation, therefore, to identify orthologous regions in vertebrates I performed sequence alignments using as query the human DNA conserved sequences. With this approach I identified orthologs throughout the mammalian and reptile and birds clade, as well as, in fishes. No \textit{libra} ortholog was identified in non-vertebrate genomes. Using the conserved block with higher sequence variability I built a Bayesian phylogenetic tree that, in agreement with evolutionary distances between species, shows fish sequences as the more distant to mammals and highlights their high accumulation of substitutions.\\

			Importantly, in all \textit{libra} orthologous regions it was possible to identify a conserved near-perfect mir-29 binding site flanked by sequences that also show high sequence conservation. In contrast, the open reading frame (ORF) that is present in mammalian \textit{libra} orthologs is not conserved in fishes, and coelacanth is the only fish which orthologous region contained a short DNA region that codes for a small 10-aa truncated version of the protein. These results indicate the emergence of \textit{libra} as a long noncoding RNA locus that gained an ORF in the last common ancestor of mammals, reptiles and birds as the most parsimonious scenario of its evolution. Finally, the high conservation of the miR-29 binding site suggest conserved regulatory functions of the miRNA-target pairing throughout vertebrates.\\

		\includepdf[pages=-, pagecommand={\thispagestyle{plain}}, scale=0.85]{NSMB_2018/s41594-018-0032-x.pdf}

		\includepdf[pages=-, pagecommand={\thispagestyle{plain}}, scale=0.9]{NSMB_2018/41594_2018_32_MOESM1_ESM_short.pdf}


