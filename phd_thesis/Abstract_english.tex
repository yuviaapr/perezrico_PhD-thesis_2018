\begin{abstract}{Abstract}

Gene regulation in vertebrates involves a multitude of regulators and cellular machineries to orchestrate complex interactions that control cellular responses and are the basis to establish cell identity. Transcriptional regulation is considered as the first layer to modulate the expression level of a gene. In order to achieve efficient transcription, promoters have to interact with enhancers. Cell identity genes tend to be associated with clusters of enhancers, referred as super-enhancers, which are highly sensitive to changes in concentration of their binding proteins and display high chromatin interconnectivity. Super-enhancers are frequently organized in the nucleus as insulated neighborhoods delimited by architectural proteins, such as, the CCCTC-binding factor (CTCF). Considering the complex networks that super-enhancers can form with their target genes, it has been hypothesized that they can participate in the formation of phase separated foci. On the other hand, CTCF has pleiotropic functions that depend on their binding sites and on the type of chromatin interactions in which it is involved. Besides conferring insulation to super-enhancer neighborhoods, CTCF is important for the insulation of structural domains in general and in the formation of DNA loops. For these reasons, both super-enhancers and CTCF are considered as central players in the genome organization that directly impact the regulation of gene expression.\\

Even though the functions of super-enhancers and CTCF have been extensively investigated in mammalian genomes, the conservation of their functions in the genomes of phylogenetically distant vertebrates from mammals has remained largely unexplored. Here, to gain insight into their functional conservation, analyses of super-enhancers and CTCF in zebrafish were performed. Super-enhancers annotated in zebrafish show high cell and tissue specificity, and the main difference identified when compared to mouse and human super-enhancers is their distribution relative to transcription start sites. Conservation analyses indicate that super-enhancers do not have higher sequence conservation than typical enhancers. Nevertheless, by restricting the analysis to those super-enhancers that are located in close proximity to orthologs in zebrafish, mouse and human, it is possible to identify a subset of super-enhancers that have higher sequence conservation than the rest. Comparison of the expression patterns driven by constitutive regions of two super-enhancers associated with orthologs enabled the identification of regions controlling similar expression patterns in spite of no evident sequence conservation. Regarding CTCF, analyses of the small fraction of CTCF binding regions that overlap promoters indicate a correlation between the abundance of CTCF and gene expression, which could be explained by blockage of nucleosome deposition at those promoters. Importantly, in contrast to the observed distribution of CTCF around contact domains in mammalian genomes, CTCF is not enriched at boundaries of domains in zebrafish embryos.\\

In summary, these results show evidence of conserved and divergent functions of gene regulators throughout vertebrate evolution and set a precedent for studies of genome organization in zebrafish. Future integration of super-enhancer annotations and CTCF binding regions will be important to analyze the contribution of these and additional regulators in this intricate process.\\

\end{abstract}



