\chapter{Discussion and perspectives}

Extensive analyses of super-enhancers and CTCF in mammalian genomes indicate that these two regulators have major roles in the 3D genome organization, however, little is known about their functions in other vertebrates. In this work, I present the first steps towards the elucidation of super-enhancer and CTCF functions in the zebrafish genome. The results confirm that super-enhancers in zebrafish are cell- and tissue-specific, indicating an association with the establishment of cell identity, and highlight the benefits of analyzing whole super-enhancer regions instead of limiting the analysis to single enhancers. Regarding the analyses of CTCF binding in the zebrafish genome, the results support the notion of a CTCF ``code’’ widespread in vertebrate genomes. Conversely, in spite of the similar genome-wide distribution of CTCF relative to transcription units in zebrafish and mammalian genomes, our results show discrepancy in the distribution of CTCF around contact domains, suggesting differences in the regulation of the 3D genome organization. Future work should aim to analyze the interplay of super-enhancers and CTCF towards the identification of determinants of the genome organization in zebrafish and potential novel regulators.\\

	\section{Comparative analyses of super-enhancers reveal conserved elements in vertebrate genomes}

		\subsection{Zebrafish super-enhancer characteristics}

Out of the previously described characteristics of super-enhancers in mammals, analysis of zebrafish super-enhancers was focused on their cell and tissue specificity and on their distribution relative to gene bodies. Whereas the high cell and tissue specificity of zebrafish super-enhancers was confirmed, differences in their distribution in the genome were observed. Contrary to mouse and human super-enhancers located in the vicinity of genes, zebrafish super-enhancers are not preferentially located downstream of the TSSs and, in general, they are not enriched over gene bodies. If this observation has an impact in the relationship between super-enhancers and their targets genes has not been evaluated. In addition, the identity of the genes with super-enhancer enrichment in mouse and human was not analyzed to assess if those genes show particular features or if they correspond to specific gene classes. Also, the reference annotations of mouse and human are more complete and curated; therefore, the analyses of zebrafish super-enhancers could have been partially hampered by missing gene annotations, including those of noncoding RNA genes.\\

It remains to be tested if other super-enhancer characteristics described in mammals also apply to zebrafish super-enhancers. Among those characteristics, it will be important to determine if zebrafish super-enhancers have high sensitivity to perturbations of their binding proteins (Whyte et al. 2013; Lovén et al. 2013; Vahedi et al. 2015), if they are also confined in insulated neighborhoods (Dowen et al. 2014) and if they have spatial colocalization (Beagrie et al. 2017) to create complex regulatory networks (Novo et al. 2018). These characteristics will be important to analyze because they can provide clues about the establishment of super-enhancers in light of the phase separation and loop extrusion models (Fudenberg et al. 2016; Hnisz et al. 2017). In fact, super-enhancers are involved in intra- and inter-chromosomal interactions in 8 and 24 hpf zebrafish embryos (Kaaij et al. 2018), indicating that super-enhancers in zebrafish might also form phase separated foci.\\

		\subsection{Analysis of super-enhancer sequence conservation}

Sequence conservation assessed by PhastCons scores showed that super-enhancers do not have higher conservation than typical enhancers, and only those super-enhancers that are associated with orthologs in zebrafish, mouse and human show significant higher sequence conservation than the rest of super-enhancers. Nevertheless, it was not analyzed if these increase in sequence conservation was a general increase of conservation in the whole super-enhancer regions, or if it was rather caused by small regions with high conservation values. It is important to analyze this difference, as small intergenic regions of high conservation could be an indicative of conserved \textit{cis}-regulatory modules. Therefore, this information could be used as a basis to identify super-enhancers containing regions driving conserved patterns of expression, similar to the regions described for SE-\textit{zic2a} and SE-\textit{irf2bpl}. Independently of their sequence conservation, the identified subset of super-enhancers with conserved association to orthologous genes could also be used to predict short regions of conserved functionally, as it has been previously performed using noncoding sequence elements with no evident conservation based on pair-wise comparisons of distant genomes (Taher et al. 2011). Notably, the scanning of the identified regions driving similar patterns of gene expression within SE-\textit{zic2a} and SE-\textit{irf2bpl} did not show any specific conserved TFBS grammar, suggesting that specialized algorithms to identify low conserved regulatory modules (Taher et al. 2011) and the scanning with more TF matrix models will be required to identify the mechanism governing their functional conservation. These analyses would have to be complemented with enhancer reporter assays bashing the regions to identify the minimal sequences driving equivalent expression patterns.\\

In addition, the process that leads to establishment of super-enhancers near orthologous genes has not been assessed. One possibility is that these super-enhancers are located in syntenic blocks in the three species, indicating a common origin. This evidence combined with annotations of contact domains can be used to analyze if super-enhancers and those orthologs are located within the same contact domain, explaining why their preferential associations have been maintained in evolution. Another possibility is that these super-enhancers emerged independently in each species, which is also a viable explanation considering that super-enhancers can emerged from single nucleation events (Mansour et al. 2014).\\

		\subsection{Improvement in the annotation of target genes}

Super-enhancer target genes were annotated based on a simple method relying on sequence proximity. Comparison of annotations based on proximity and ChIA-PET data have shown good correlation (Dowen et al. 2014). Nevertheless, it is clear that annotations based on chromosome conformation capture data are more reliable because enhancers can control the expression of target genes that are located more distantly than 100 kb (Sanyal et al. 2012), the proximity value used to annotate target genes. Considering that adjacent super-enhancers and genes could be located in different contact domains it is likely that, besides disregarding true target genes, annotations solely based on proximity potentially include false positive target genes.\\

		\subsection{Identification of TFs governing super-enhancer functions}

Although zebrafish ATAC-seq data was available for the dome stage, the performed enhancer reporter assays and comparison with nanog ChIP-seq data indicate that ATAC-seq is not sufficient to comprehensively identify TFBS epicenters. Therefore, experimental data sets interrogating the binding of more TFs are required to better characterize the TFs enriched at super-enhancers and their cooperativity within them. Despite the lack of antibodies with reactivity to zebrafish, determination of TF binding in the genome can now be achieved using CRISPR/Cas9, as shown for CTCF.\\

Differential analysis of ATAC-seq regions within typical enhancers and super-enhancers identified motifs showing similarity to SOX and ESRRA matrix models, however, experimental validation is necessary to confirm if there is a particular collaboration of these factors within super-enhancers. For instance, binding of these transcription factors to the composite motifs could be confirmed by electrophoretic mobility shift assays, whereas, sequential ChIP-seq could be performed to identify co-bound regions and assess their enrichment at super-enhancers.\\

		\subsection{Biological relevance of super-enhancers}

After the characterization of super-enhancers in mouse and human cells and tissues (Whyte et al. 2013; Lovén et al. 2013; Hnisz et al. 2013) their biological relevance started to be questioned, as there is the assumption that super-enhancers have to act as a single unit to be considered as regulators apart of what is considered an enhancer. Most of this criticism is supported on genetic deletions performed in super-enhancers showing that some of their constituent enhancers act independently and in additive manner, as expected of enhancers (Hay et al. 2016; Moorthy et al. 2017). These results have then minimized the value of identifying super-enhancers. However, genetic deletions have also shown that super-enhancers can in fact act in a synergistic manner, and that deletion of small regions can lead to the collapse of the whole super-enhancer or interfere with the regulation of target genes (Mansour et al. 2014; Shin et al. 2016). Therefore, it is not reasonable to conclude that super-enhancers do not represent a particular subset of regulators enhancing gene expression just because their mechanisms of action cannot be generalized, as even the rules governing enhancer and promoter functions cannot be generalized. In addition, the fact that some genes involved in the establishment of cell identity are not associated with a super-enhancer (Moorthy et al. 2017) do not discredit that super-enhancers are at the core of transcriptional networks (Hnisz et al. 2015; Lin et al. 2016; Saint-André et al. 2016).\\

However, it is important to emphasize that the current process to identify super-enhancers is only based on ChIP-seq signal and the threshold to consider a stitched region as super-enhancer is the inflection point. For this reason, results of the ROSE program to identify super-enhancers have to be taken cautiously and all the characteristics described for super-enhancers cannot be directly attributed to the regions identified as super-enhancers without performing analyses to evaluate their characteristics. This is particularly relevant for the super-enhancers with ChIP-seq abundances slightly higher than the threshold, but as a general rule, a super-enhancer cannot be considered as important for cell fate without experimental data that supports this conclusion. I consider that identification of super-enhancers is a strong strategy to start to investigate gene regulation in species, or samples in general, for which few genomic data sets are available. Predictions of potential master regulators obtained from the study of super-enhancers could then be used to design additional experiments and improve the annotation of regulators. I envision that the annotation of super-enhancers could be improved by incorporating Hi-C data or data generated by other variants of 3C-based techniques that interrogate chromatin interactions genome-wide. Currently, it is known that regions that can be considered to act as \textit{bona fide} super-enhancers, in the sense that they represent regions of high interconnectivity to exert their function, are visualized in Hi-C maps as regions with high frequency of interaction that often appear as ``stripes’’ (Vian et al. 2018). Therefore, one approach to identify better quality super-enhancers will be to first annotate super-enhancers using the ROSE program and then calculate the intra-interaction frequency of these regions to filter out those with low interaction frequencies.\\

The selection of super-enhancers to dissect their functions by enhancer reporter assays was based on their identification in pluripotent cells and brain tissues and in the known functions of Zic2 (Luo et al. 2015). However, these two super-enhancers were also ranked as top super-enhancers by ROSE, thus, it is likely that this non-selected characteristic was also important to enable the identification of zebrafish and mouse regions driving similar expression patterns. For this reason, it will be interesting to assess if super-enhancers associated with orthologous genes in the three species tend to be ranked among the top super-enhancers and if this is the case, it would also support that additional criteria besides an inflection point are useful to annotate super-enhancers.\\

Enhancer reporter assays performed for the two selected super-enhancers also led to the conclusion that analysis of whole super-enhancer regions are useful to study enhancer redundancy, considering that possible ``shadow’’ enhancers could be contained within these regions of active chromatin to facilitate their activation under specific conditions. Another important conclusion of the analyses here presented is that conserved association with orthologous genes can be used as a criterion to select super-enhancers for their further characterization to assess their roles in a specific tissue or cell type. Strikingly, the functions of \textit{irf2bpl} were unknown at the time that the SE-\textit{irf2bpl} was analyzed. However, recent evidence indicates that mutations in \textit{IRF2BPL} can cause neurological defects in humans (Tran Mau-Them et al. 2018). This supports the idea that the same criterion of association can also be used to select genes for additional investigation of their functions.\\

Altogether, the results obtained from the analyses of zebrafish super-enhancers in combination with the current knowledge about their implication in the control of gene expression (Whyte et al. 2013; Lovén et al. 2013; Dowen et al. 2014), processing of RNAs (Suzuki et al. 2017) and in the 3D genome organization (Beagrie et al. 2017; Vian et al. 2018; Novo et al. 2018) confirm that the study of super-enhancers is important to have a comprehensive vision of gene regulation. However, I consider that the knowledge that we have gained during the last years should be applied to improve the annotation of super-enhancers and that super-enhancers have to be substantially analyzed before implicating them in specific functions.\\

		\subsection{Impact of tissue heterogeneity in super-enhancer annotation}

The main disadvantage of the super-enhancer annotations here presented, with the exception of those of pluripotent cells, is that the H3K27ac ChIP-seq libraries that were used for the annotation were prepared by homogenization of whole tissues. This implies that the H3K27ac signal that is identified represents the average in the population of different cell types. Therefore, it is possible that a fraction of the annotated super-enhancers are false positives caused by the effect of merging single enhancers located around the same loci in different cell types. In particular, one of the previously defined super-enhancers associated with \textit{Myc} in mouse do represent a cluster of enhancers, but these enhancers act in the hierarchy of hematopoietic lineages rather than in a specific cell type (Bahr et al. 2018). Hence, selection of specific cell types before preparing sequencing libraries will provide a refined set of super-enhancers. For instance, fluorescence activated cell sorting has been coupled to ATAC-seq to generate annotations of endothelial specific enhancers in zebrafish (Quillien et al. 2017). Importantly, a broad set of lineage specific markers is now available (Wagner et al. 2018; Farrell et al. 2018) and will be fundamental in future analyses of gene regulation.\\

	\section{\textit{In vivo} analysis of CTCF functions in the zebrafish genome}

		\subsection{Transgenic lines to overcome the lack of antibodies in zebrafish}

In spite of being a well-established model organism, only few TF ChIP-seq libraries of zebrafish have been generated (Xu et al. 2012; Nelson et al. 2014, 2017; Meier et al. 2018). This can be partially explained by the fact that few antibodies against TFs have reactivity to zebrafish. Despite that two independent groups have developed CTCF custom antibodies, (Delgado-Olguín et al. 2011; Meier et al. 2018; Carmona-Aldana et al. 2018) they have not been used to generate ChIP-seq libraries, even when these data would have strength the claims of these analyses. Hence, to avoid problems with custom antibodies that might not have the needed quality to perform ChIP, I raised a zebrafish line with a tagged version of CTCF and obtained good quality ChIP-seq data sets. This was possible by applying a strategy to tag the endogenous protein using the CRISPR/Cas9 system and it will prove useful to analyze the functions of more proteins for which antibodies are not available. This is particularly important because notwithstanding the fact that motifs obtained with mammalian data can be successfully used to predict TFBSs in the zebrafish genome, we showed that the zebrafish CTCF specific core motif identifies more CTCF peaks containing a CTCF site.\\

The \textit{ctcf\textsuperscript{HA/HA}} zebrafish show normal development and reach adulthood, however, it cannot be completely discarded that the insertion of the HA-tag in the N-terminus of CTCF does not disturb the interaction between CTCF and partner proteins. One possible way to test if this insertion destabilize interactions is by generating an additional zebrafish line with CTCF tagged at the C-terminus, and compare the proteins that are co-purified with CTCF.\\

		\subsection{Functions of CTCF at promoters and intragenic regions}

Comparison of CTCF abundance at promoter regions and gene expression levels of the genes corresponding to those promoters showed that, in average, genes with high abundance of CTCF at promoters have higher expression levels than those with no or low abundance CTCF-bound promoters. This could be explained by at least two mechanisms. First, it has been previously shown that CTCF binding can stabilize enhancer-promoter interactions (Ren et al. 2017). However, with the resolution of the zebrafish Hi-C data used, it will be impossible to globally assign specific promoters to enhancers and analyze if this is the mechanism by which CTCF is facilitating the expression of those genes. Ultra-deep Hi-C analyses during neural differentiation have shown that interactions between enhancers and promoters are not strongly dependent on CTCF (Bonev et al. 2017); therefore, this evidence disfavors the mechanism of enhancer-promoter interaction stabilization in zebrafish. Second, an alternative mechanism is that CTCF participates in the establishment of nucleosome-free regions at these promoters. Indeed, ATAC-seq data indicates that the promoters with the highest enrichment of CTCF also coincide with higher accessibility signal, supporting this mechanism. In agreement with these results, genes that have early downregulation after CTCF depletion in mESCs also have high accessibility at their promoters, which are bound by CTCF (Nora et al. 2017). Although the mechanisms involving nucleosome blocking by CTCF is supported by the current data, it will be interesting to analyze if CTCF is additionally participating in the stabilization of interactions with enhancers. Hi-C might not be the more appropriate method to try to identify more defined interactions between enhancers and promoters in zebrafish given the redundancy of the genome, but the availability of the zebrafish line with the tagged version of CTCF enables the generation of CTCF HiChIP libraries. These data will be useful to directly identify CTCF-mediated loops and test this mechanism. Moreover, CTCF loops could also be used to analyze if CTCF binding events at intragenic regions are engaged in loops with the promoter to control exon inclusion as determined for humans (Shukla et al. 2011; Ruiz-Velasco et al. 2017).\\

		\subsection{Impact of mitotic cells on Hi-C contact maps}

Visualization of the Hi-C data generated with 24 hpf embryos shows a distribution of the signal that is not expected from cells in interphase. In these maps there is a second diagonal with enriched interaction frequencies that intersects at the centromere the strong diagonal corresponding to the interactions at short distances. Also, there are enriched interactions between the centromeres and telomeres. These characteristics indicate a Rabl configuration of the chromosomes, which implies that at least a fraction of the cells used to build the Hi-C libraries were undergoing mitosis. Indeed, the Rabl configuration is more evident in the contact maps of 4 hpf embryos, which are still undergoing fast rounds of mitosis.\\

In the original work for which this zebrafish Hi-C data was generated, the authors performed Hi-C at different developmental stages to analyze the dynamics of the 3D genome organization before and after the maternal to zygotic transition (Kaaij et al. 2018). They conclude that at the early blastula stage, contact and compartment domains are present, but they are loss around the transition phase and gradually gained until they become more insulated at 24 hpf. Nevertheless, the fact that they did not use synchronous embryos for the analyses limits the possibility to distinguish between cell cycle effects and stage-specific effects. Although the authors claimed that cells at the time points analyzed are not mainly going through mitosis using four DAPI stains of other matching samples, the Hi-C contact maps reflect that the 4 hpf embryos were enriched on mitotic cells, which is in agreement with the fast divisions occurring at this stage. In conclusion, experiments considering the cell cycle will have to be performed using synchronous embryos to address the dynamics of the 3D genome organization during the genome activation, as previously done in \textit{Drosophila} and mouse embryos (Hug et al. 2017; Ke et al. 2017).\\

If we consider previous Hi-C contact maps of mitotic mouse and human cells an interesting question can be derived: why the strong second diagonal is not visible in the maps of mitotic mouse and human cells? This discrepancy has a different explanation for each species. In the case of mouse, the chromosomes are telocentric and they are generally masked for analyses, therefore, the main visual evidence of mitotic cells is the observation of inter-chromosomal interactions along the whole chromosomes (Nagano et al. 2017), which is the same that can be observed for chromosome arms in zebrafish. In the case of human, chromosomes are not telocentric, but the cells used to generate human Hi-C data during mitosis were synchronized by nocodazol treatment (Naumova et al. 2013). In consequence, the spindle is not assembled and the chromosomes are not been pulled, as such, the maps do not reflect a Rabl configuration. However, if Hi-C maps were generated using cells in anaphase, the chromosomes should have a Rabl configuration.\\

		\subsection{What can be concluded about CTCF functions?}

In agreement with CTCF binding in other vertebrate genomes, CTCF can bind to extended motifs in the zebrafish genome. In this first analysis, I focused only on the upstream motif that was originally reported in mammals (Filippova et al. 1996; Rhee and Pugh 2011; Schmidt et al. 2012). However, it remains to be determined if the described downstream motif is also present in zebrafish (Ren et al. 2017). In addition, it will be interesting to determine if CTCF has different functions according to the sites that it recognizes, which is the reasoning behind the CTCF code.\\

Strikingly, enrichment of CTCF at boundaries of contact domains is not detected at 24 hpf in zebrafish embryos. This difference in enrichment compared to mammals could be originated by divergence in the relevance of CTCF functions in the 3D genome organization in zebrafish and mammals. Additional Hi-C and CTCF ChIP-seq data from more zebrafish samples is required to confirm this difference in the distribution relative to contact domains. For those analyses it would be required to use fully differentiated cell types or tissue samples mainly composed by cells in interphase, to discard the possibility that CTCF enrichment at contact domains in zebrafish was hidden by the inclusion of mitotic cells in the sample, considering that CTCF is globally removed from chromatin during mitosis (Agarwal et al. 2017). Although CTCF is not generally enriched at boundaries, a fraction of them contain CTCF peaks; therefore, it remains to be analyzed if the motifs located in these \textit{in vivo} identified regions have the preferred orientation observed in mammals (Rao et al. 2014; Guo et al. 2015).\\

In contrast to CTCF, active chromatin marks are enriched at contact domain boundaries, suggesting that transcription could be an important determinant of their establishment in zebrafish. Indeed, it has been demonstrated that transcription is the main regulatory mechanism in the formation of structural domains in \textit{Drosophila} and other eukaryotes (Rowley et al. 2017). As such, the impact of transcription in the formation of contact domains in zebrafish will be an important aspect to evaluate, this could be partially achieved by performing modelling of Hi-C contact maps using RNA-seq and histone ChIP-seq data of the marks enriched at boundaries. Inclusion of CTCF ChIP-seq data in this modelling approach could help to discern the impact of transcription and CTCF on the genome organization in zebrafish, similar to what has been shown using human data (Rowley et al. 2017).\\

		\subsection{Limitations of the analyses}

As in the case of the super-enhancer analyses, the main limitation to characterize CTCF functions in embryos is the heterogeneity of the sample. This was particularly evidenced when I performed the analysis of promoters marked by H3K4me3 and H3K27me3, because it is not possible to distinguish if the signal was actually enriched in the same promoter. Furthermore, as it has already been discussed above, the resolution of the Hi-C data is a limiting factor to identify loops between specific regions. Hence, CTCF HiChIP in specific cell types represents an effective solution to overcome both limitations.\\

In conclusion, in this work I have analyzed the characteristics and functions of super-enhancers and CTCF in zebrafish. The results here obtained will be valuable to design strategies to further evaluate the functions of these regulators and their impact on the 3D genome organization. Finally, the integration of super-enhancer annotations and CTCF data will be important to predict additional regulators in the zebrafish genome and assess their conservation.\\



